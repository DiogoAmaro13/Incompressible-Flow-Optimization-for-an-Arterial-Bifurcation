\documentclass[hidelinks]{article}
\usepackage[a4paper, total={7in, 10in}]{geometry}
\usepackage[dvipsnames]{xcolor}
\usepackage{amsmath}
\usepackage{tikz}
\usepackage{tkz-euclide}
\usepackage[unicode]{hyperref}
\usepackage[all]{hypcap}
\usepackage{fancyhdr}
\usepackage{amsfonts}
\usepackage{float}
\usepackage{subcaption}
\usepackage{caption}

\usetikzlibrary{angles,calc, decorations.pathreplacing}

\definecolor{carminered}{rgb}{1.0, 0.0, 0.22}
\definecolor{capri}{rgb}{0.0, 0.75, 1.0}
\definecolor{brightlavender}{rgb}{0.75, 0.58, 0.89}

\begin{document}

We want to solve the optimal control problem given by $$\text{min}\: J(v,p,u)=\frac{1}{2} \int_{\Gamma_{obs}} |v-v_d|^2ds + \frac{\alpha_1}{2}\int_{\Gamma_C} |\nabla_t u|^2ds+\frac{\alpha_2}{2}\int_{\Gamma_C} |u|^2ds$$
s.t.
\begin{equation}\label{stokes}
    \begin{cases}
- \nu \Delta v + \nabla p = f       & \text{in } \Omega\\
             \text{div} v = 0       & \text{in } \Omega\\
                        v = g       & \text{on } \Gamma_{in}\\
                        v = 0       & \text{on } \Gamma_{w}\\
   p n - \nu \partial_n v = u       & \text{on } \Gamma_{C}
\end{cases}.
\end{equation}\newline


We want to solve this configuration by considering a 2D realistic representation of an arterial bifurcation, parametrized as shown in Fig.~\ref{domain}.
We consider an inverse problem in hemodynamics, focusing on a simplified model of an arterial bifurcation. The computational domain is parametrized, with an inflow boundary denoted as \( \Gamma_{\text{in}} \), two outflow boundaries \( \Gamma_C \), and the physical vessel wall represented by \( \Gamma_D \). The primary variables of interest are the velocity $\vec{v}$ and pressure $p$, which are assumed to satisfy~\eqref{stokes}.

\begin{figure}[H]\label{domain}
     \centering
     \begin{subfigure}[t]{0.45\textwidth}
         \centering
         \includegraphics[width=\textwidth]{assets/bif.pdf}
         \caption{Boundary conditions: no-slip conditions on $\Gamma_D(\mu)$, Poiseuille velocity profile $g(\mu_{in})$ on $\Gamma_{in}$, unknown Neumann flux on the outflow sections.}
     \end{subfigure}
     \begin{subfigure}[t]{0.45\textwidth}
         \centering
         \includegraphics[width=\textwidth]{assets/mu_param.pdf}
         \caption{Parametrization of the original domain}
     \end{subfigure}
    \caption{2D representation of the arterial bifurcation considered for this implementation.}
\end{figure}

We assume that the velocity profile is known along a segment of the inflow boundary \( \Gamma_{\text{in}} \), but no direct measurements of the Neumann flux at the outflow boundaries \( \Gamma_C \) are available. The control variable in this problem is the unknown Neumann flux at \( \Gamma_C \). The goal is to deduce this control variable using the velocity data along \( \Gamma_{\text{in}} \), and consequently, to recover the velocity and pressure fields across the entire domain.

The problem involves several parameters, including geometric parameters \( \mu_{\text{geom}} \), which describe the dimensions of the bifurcation (e.g., the length of each branch, the angle of the bifurcation), a parametrized velocity profile \( \mu_{\text{meas}} \), and a parametrized inflow velocity profile \( g(\mu_{\text{in}}) \). These parameters together influence the flow field that we aim to reconstruct. This formulation represents a simplified version of the more complex hemodynamic flow problem, assuming two-dimensional geometry, steady-state conditions, and simplified constitutive laws. We further assume that the measured velocity profile can be approximated by a simple analytical function, parametrized for the data assimilation process.

\end{document}